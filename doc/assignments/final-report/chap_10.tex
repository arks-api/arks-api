\chapter{ Deployment and Maintenance}
Software deployment is all of the activities that make a software system available for use. The general deployment process consists of several interrelated activities with possible transitions between them. These activities can occur at the producer site or at the consumer site or both. Because every software system is unique, the precise processes or procedures within each activity can hardly be defined. Therefore, "deployment" should be interpreted as a general process that has to be customized according to specific requirements or characteristics. A brief description of each activity will be presented later.
\paragraph{}
We are deploying our cloud server in our college campus and the client application is deployed over  internet so that anyone can download the application and use the services of the cloud. Note that, this cloud is deployed in the college campus only so it is not accessible from outside of college.
\paragraph{}
\section{Installation and Uninstallation}
\subsubsection{A. Sun JDK 1.7 Installation}
\begin{enumerate}
\item Double click jdk-7-ea-bin-b31-windows-i586 to run the installation program. JDK license dialog displayed. Accept the license in order to install JDK.
\item The JRE Custom setup dialog enables you to choose a custom directory for JRE files.
\item The complete dialog indicates a successful installation.
\end{enumerate}
\paragraph{}
\subsubsection{B. Netbeans-7.3 Installation}
\begin{enumerate}
\item Click Netbeans-7.0.1.exe to run the installation program. Netbeans IDE, Installer  dialog displayed. Click next to continue installation.
\item Tick the license agreement and click next.
\item Choose the path for NetBeans, JDK installation and click next.
\item Click Install and the installation starts.
\item Click Finish and the installation completes successfully.
\end{enumerate}
\paragraph{}
\section{User Help}
The user will just have to browse and select the image and provide the required processing which will generate OCR text.\\