\chapter{ Introduction}
\paragraph{} Basic project idea is to reduce manual efforts for content based searching in large set of documents using Hadoop Big Data Management framework to automate content based searching and retrieving the relevant content.

\section{Overview}
\paragraph{} Taks like assignment completion, taking notes from text books and reference books on particular topic, topics for presentation need deep reading and need to go through every document manually just to find relevant content on given topic.Currently present systems are only searching based on document title, author, size, and time but not on content. So to do content based search on big data documents and large text data Haddop framework can be used.
\paragraph{} So using Hadoop Big Data management framework consist of HDFS, MapReduce, and HBase, we are developing content based search on PDF documents to solve real life problem. So this is basic motivation for the project.

\subsection{Brief Introduction}
\paragraph{}
Unstructured data like doc, pdf, accdb is lengthy to search and filter for desired information. We need to go through every file manually for finding information. It is very time consuming and frustrating. It doesn’t need to be done this way if we can use high computing power to achieve much faster content retrieval.
\paragraph{} We can use features of big data management system like Hadoop to organize unstructured data dynamically and return desired information. Hadoop provides features like Map Reduce, HDFS, HBase to filter data as per user input. Finally we can develop Hadoop Addon for content search and filtering on unstructured data. This addon will be able to provide APIs for different search results and able to download full file, part of files which are actually related to that topic. It will also provide API for context aware search results like most visited documents and much more relevant documents placed first so user work get simplified.
\paragraph{} This Addon can be used by other industries and government authorities to use Hadoop for their data retrieval as per their requirement.
\paragraph{} After this addon, we are also planning to add more API features like content retrieval from scanned documents and image based documents.

\subsection{Problem Definition}
Hadoop add-on API for Advanced Content Based Search \& Retrieval \\
To find the relevant content from the huge number of PDF files present on Hadoop Distributed File System

\subsection{Applying software engineering approach}
Waterfall approach was first Process Model to be introduced and followed widely in software engineering to ensure success of the project. In the waterfall approach, the whole process of software development is divided into separate process phases. The phases in Waterfall model are: Requirement Specifications phase, Software Design, Implementation and Testing and Maintenance. All these phases are cascaded to each other so that second phase is started as and when defined set of goals are achieved for first phase and it is signed off, so the name waterfall model \\

Steps performed in waterfall model are;
\subsubsection{1) Requirement Gathering and Analysis}
All possible requirements of the system to be developed are captured in this phase. Requirements are set of functionalities and constraints that the end-user (who will be using the system) expects from the system. The requirements are gathered from the end-user by consultation, these requirements are analyzed for their validity and the possibility of incorporating the requirements in the system to be development is also studied. Finally, a Requirement Specification document is created which serves the purpose of guideline for the next phase of the model.\\
\subsubsection{2) System and Software Design}
Before a starting for actual coding, it is highly important to understand what we are going to create and what it should look like? The requirement specifications from first phase are studied in this phase and system design is prepared. System Design helps in specifying hardware and system requirements and also helps in defining overall system architecture. The system design specifications serve as input for the next phase of the model.\\
\subsubsection{3) Implementation}
On receiving system design documents, the works divided in modules/units and actual coding is started. The system is first developed in small programs called units, which are integrated in the next phase. Each unit is developed and tested for its functionality; this is referred to as Unit Testing. Unit testing mainly verifies if the modules/units meet their specifications.\\
\subsubsection{4) Testing}
As specified above, the system is first divided in units which are developed and tested for their functionalities. These units are integrated into a complete system during Integration phase and tested to check if all modules/units coordinate between each other and the system as a whole behaves as per the specifications. Also testing involves black box and white box testing which will be useful in identifying the bugs in functionality and code of the system. After successfully testing the software, it is delivered to the customer.