\chapter{Technical Specifications}
\section{Technology Details}
\subsection{Java}
Java is a general-purpose, concurrent, class-based, object-oriented computer programming language that is specifically designed to have as few implementation dependencies as possible. It is intended to let application developers "write once, run anywhere" (WORA), meaning that code that runs on one platform does not need to be recompiled to run on another. Java applications are typically compiled to byte code (class file) that can run on any Java virtual machine (JVM) regardless of computer architecture. Java is, as of 2012, one of the most popular programming languages in use, particularly for client-server web applications, with a reported 10 million users.
\paragraph{}
Java was originally developed by James Gosling at Sun Microsystems (which has since merged into Oracle Corporation) and released in 1995 as a core component of Sun Microsystems' Java platform. The language derives much of its syntax from C and C++, but it has fewer low-level facilities than either of them. The original and reference implementation Java compilers, virtual machines, and class libraries were developed by Sun from 1991 and first released in 1995. As of May 2007, in compliance with the specifications of the Java Community Process, Sun relicensed most of its Java technologies under the GNU General Public License. Others have also developed alternative implementations of these Sun technologies, such as the GNU Compiler for Java and GNU Class path.

\subsubsection{Features of Java}
\begin{enumerate}
\item Platform Independent
\item Simple
\item Object Oriented
\item Robust
\item Distributed
\item Portable
\item Dynamic
\item Secure
\item Multithreaded
\item Architectural Neutral
\end{enumerate}

\paragraph{}
Java platform is the name for a bundle of related programs from Sun that allow for developing and running programs written in the Java programming language. The platform is not specific to any one processor or operating system, but rather an execution engine (called a virtual machine) and a compiler with a set of libraries that are implemented for various hardware and operating systems so that Java programs can run identically on all of them.